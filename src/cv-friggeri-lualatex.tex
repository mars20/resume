%%%%%%%%%%%%%%%%%%%%%%%%%%%%%%%%%%%%%%%%
% Friggeri Resume/CV
% XeLaTeX Template
% Version 1.2 (3/5/15)
%
% This template has been downloaded from:
% http://www.LaTeXTemplates.com
%
% Original author:
% Adrien Friggeri (adrien@friggeri.net)
% https://github.com/afriggeri/CV
%
% License:
% CC BY-NC-SA 3.0 (http://creativecommons.org/licenses/by-nc-sa/3.0/)
%
% Important notes:
% This template needs to be compiled with XeLaTeX and the bibliography, if used,
% needs to be compiled with biber rather than bibtex.
%
%%%%%%%%%%%%%%%%%%%%%%%%%%%%%%%%%%%%%%%%%

\documentclass[print]{friggeri-cv} % Add 'print' as an option into the square bracket to remove colors from this template for printing

\hypersetup {
  pdfauthor={Schuyler Eldridge},
  pdftitle={Schuyler Eldridge -- CV},
}

\addbibresource{schuyler.bib} % Specify the bibliography file to include publications

\usepackage{xspace}

\begin{document}

\header{Schuyler}{ Eldridge}{computer engineer} % Your name and current job title/field

%----------------------------------------------------------------------------------------
%	SIDEBAR SECTION
%----------------------------------------------------------------------------------------

\newcommand{\CPP}
{C\nolinebreak[4]\hspace{-.05em}\raisebox{.22ex}{\footnotesize ++}\xspace}

\begin{aside} % In the aside, each new line forces a line break
\section{contact}
\href{mailto:schuye@bu.edu}{schuye@bu.edu}
\href{tel:+19143821315}{+1 (914) 382 1315}
US Citizen
~
\href{https://seldridge.github.io}{seldridge.github.io}
\href{https://github.com/seldridge}{github.com/seldridge}
\section{hardware}
Chisel
Verilog/SystemVerilog
\section{software}
Assembly (x86, RISC-V)
Bash
C/\CPP
GNU Make
Java
\LaTeX
Matlab
Perl
Python
Scala
TCL
\section{tools}
Cadence RTL Compiler
Cadence SoC Encounter
Emacs
Gem5
Git
GTKWave
Icarus Verilog
Jenkins
Linux
Modelsim
Xilinx ISE/Vivado
\section{coursework}
Advanced Data Structures
Computer Architecture
Control Systems
Digital Design in Verilog
Digital Signal Processing
High Performance Programming
Microprocessors
VLSI
\end{aside}

%----------------------------------------------------------------------------------------
%	EDUCATION SECTION
%------------------------------------------------ ----------------------------------------

\section{education}

\begin{entrylist}

%------------------------------------------------

\entry
{2010--2016}
{Ph.D {\normalfont Computer Engineering}}
{Boston University}
{Thesis: \emph{Viable Neuromorphic Computing via Neural Network Accelerators} \\
  Design, implementation, and management of neural network accelerators for general use, approximate computing, and prediction of microprocessor state
  \begin{itemize}
  \item Designed a fixed topology neural network accelerator for mathematical function approximation
    \begin{itemize}
    \item Implemented in \textbf{Verilog}
    \item Evaluated for energy-efficiency with a \textbf{Cadence} ASIC tooflow
    \end{itemize}
  \item Designed a multi-transaction, arbitrary topology neural network accelerator
    \begin{itemize}
    \item Modeled in \textbf{\CPP} and integrated and tested with \textbf{gem5}
    \item Implemented in both \textbf{SystemVerilog} and \textbf{Chisel}
    \item Integrated as a coprocessor of a \textbf{RISC-V} microprocessor
    \item Interfaced with user and kernel via libraries in \textbf{C} and \textbf{RISC-V Assembly}
    \item Evaluated on a \textbf{Xilinx FPGA} platform
    \end{itemize}
  \item Setup and managed an automated testing server running \textbf{Jenkins}
  \end{itemize}
  \emph{Biologically-Inspired Optical Flow on FPGA} \\
  Implementation of a biologically inspired optical flow algorithm on FPGA
  \begin{itemize}
  \item Reverse engineered a camera interface to capture consecutive frames of real-time video
  \item Computed optical flow by shifting one image with respect to the other, filtering both with a Gabor filter bank, and correlating the results
  \item Designed linear interpolation, square root, division, and CORDIC units as well as associated state machines in \textbf{Verilog}
  \item Wrote a custom \textbf{Verilog} UART transmitter/receiver and interfaced this system with \textbf{Matlab}
  \end{itemize}
}

%------------------------------------------------

\entry
{2006--2010}
{BS {\normalfont Electrical Engineering}}
{Boston University}
{Summa Cum Laude -- GPA 3.85/4.0
  \begin{itemize}
  \item Senior Design Project: PID-controlled Object Tracking System on FPGA and DSP
  \item \href{https://www.youtube.com/watch?v=S2LgUL5JLqQ}{Frogger clone in \textbf{Verilog} running on FPGA}
  \item 32-bit MIPS 5-stage pipelined CPU in \textbf{Verilog}
  \end{itemize}
}

%------------------------------------------------

\end{entrylist}

%----------------------------------------------------------------------------------------
%	WORK EXPERIENCE SECTION
%----------------------------------------------------------------------------------------

\section{experience}

%% \subsection{Internships}

\begin{entrylist}

%------------------------------------------------

\entry
{Summers 2013--2015}
{NASA Jet Propulsion Lab}
{Pasadena, CA}
{\emph{Space Technology Research Fellow}\\
  Performed neural network accelerator research focusing on their potential for fault tolerance (2013), design of a neural network accelerator simulator in \textbf{\CPP} (2014), and migration of a \textbf{SystemVerilog} hardware implementation to \textbf{Chisel} (2015)}

\entry
{Summer 2011}
{Intel Corporation}
{Hudson, MA}
{\emph{Graduate Technical Intern} \\
  Wrote tests to verify the functionality of memory controllers in \textbf{SystemVerilog} as part of the Xeon server group
}

\entry
{Summer 2010}
{Intel Corporation}
{Hudson, MA}
{\emph{Graduate Technical Intern} \\
  Designed and optimized hashing functions in \textbf{x86 assembly} while evaluating functions for quality
}

%------------------------------------------------

\end{entrylist}

%----------------------------------------------------------------------------------------
%	AWARDS SECTION
%----------------------------------------------------------------------------------------
\clearpage
\newgeometry{left=1.5cm,top=1.5cm,right=1.5cm,bottom=1.5cm,nohead,nofoot}
%% \newgeometry{left=1.5cm}

\section{awards}

\begin{entrylist}

%------------------------------------------------

\entry
{2012--2016}
{Space Technology Research Fellowship}
{NASA}
{Four year NASA fellowship titled \emph{Biologically-Inspired Hardware for Land/Aerial Robots}. This award has included yearly on-sites at NASA facilities.}

\entry
{2012}
{CELEST/CompNet Award}
{Boston University}
{Awarded at Boston University's Science Day for work titled \emph{Biologically-Inspired Hardware for Autonomous Robots}}

\entry
{2010}
{Boston University Dean's Fellowship}
{Boston University}
{Full scholarship merit award for first year Ph.D students}

\entry
{2010}
{P. T. Hsu Memorial Award for Outstanding Senior Design Project}
{Boston University}
{Awarded to the best senior design project team}

\entry
{2006--2010}
{Boston University Engineering Scholar Award}
{Boston University}
{Undergraduate merit award with half-tuition scholarship}

%------------------------------------------------

\end{entrylist}

%----------------------------------------------------------------------------------------
%	INTERESTS SECTION
%----------------------------------------------------------------------------------------

%% \section{interests}

\section{interests}
Former nationally competitive figure skater
\begin{itemize}
\item 2007 US Nationals competitor
\item 2009 and 2010 US National Intercollegiate Team Champion representing Boston University
\end{itemize}

%% \textbf{professional:} hardware design, machine learning, open source software\\ \textbf{personal:} figure skating, skiing, biking, piano

%----------------------------------------------------------------------------------------
%	PUBLICATIONS SECTION
%----------------------------------------------------------------------------------------

\section{publications}

\printbibsection{inproceedings}{conference and workshop publications}

\printbibsection{article}{articles}

\printbibsection{report}{tech reports}

\printbibsection{misc}{patents and patent applications}

%----------------------------------------------------------------------------------------

\end{document}
