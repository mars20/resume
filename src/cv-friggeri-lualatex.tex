%%%%%%%%%%%%%%%%%%%%%%%%%%%%%%%%%%%%%%%%%
% Friggeri Resume/CV
% XeLaTeX Template
% Version 1.2 (3/5/15)
%
% This template has been downloaded from:
% http://www.LaTeXTemplates.com
%
% Original author:
% Adrien Friggeri (adrien@friggeri.net)
% https://github.com/afriggeri/CV
%
% License:
% CC BY-NC-SA 3.0 (http://creativecommons.org/licenses/by-nc-sa/3.0/)
%
% Important notes:
% This template needs to be compiled with XeLaTeX and the bibliography, if used,
% needs to be compiled with biber rather than bibtex.
%
%%%%%%%%%%%%%%%%%%%%%%%%%%%%%%%%%%%%%%%%%

\documentclass[]{friggeri-cv} % Add 'print' as an option into the square bracket to remove colors from this template for printing

\hypersetup {
  pdfauthor={Schuyler Eldridge},
  pdftitle={Schuyler Eldridge -- CV},
}

\addbibresource{schuyler.bib} % Specify the bibliography file to include publications

\begin{document}

\header{Schuyler}{ Eldridge}{computer engineer} % Your name and current job title/field

%----------------------------------------------------------------------------------------
%	SIDEBAR SECTION
%----------------------------------------------------------------------------------------

\newcommand{\CPP}
{C\nolinebreak[4]\hspace{-.05em}\raisebox{.22ex}{\footnotesize\bf ++}}

\begin{aside} % In the aside, each new line forces a line break
\section{contact}
\href{mailto:schuye@bu.edu}{schuye@bu.edu}
\href{tel:+19143821315}{+1 (914) 382 1315}
US Citizen
~
\href{https://seldridge.github.io}{seldridge.github.io}
\href{https://github.com/seldridge}{github://seldridge}
\section{hardware}
Verilog
System Verilog
Chisel
\section{software}
Assembly (x86, RISC-V)
Bash
C/\CPP
GNU Make
Java
\LaTeX
Matlab
Perl
Python
Scala
TCL
\section{tools}
Cadence RTL Compiler
Cadence SoC Encounter
Emacs
Gem5
GTKWave
Git
Icarus Verilog
Linux
Modelsim
\section{coursework}
Advanced Data Structures
Computer Architecture
Control Systems
Digital Design in Verilog
Digital Signal Processing
Microprocessors
VLSI
\end{aside}

%----------------------------------------------------------------------------------------
%	EDUCATION SECTION
%----------------------------------------------------------------------------------------

\section{education}

\begin{entrylist}

%------------------------------------------------

\entry
{2010--2016}
{Ph.D {\normalfont Computer Engineering}}
{Boston University}
{\emph{Tightly Integrated Neural Network Accelerator Architectures} \\
  Design, implementation, and management of neural network accelerator architectures for approximate computing, general use, and prediction of microprocessor state
  \begin{itemize}
  \item Parameterized fixed-topology neural network accelerators in Verilog as mathematical function approximators
  \item Multi-transaction, arbitrary-topology neural network accelerator for a RISC-V microprocessor modeled in C++ and implemented in SystemVerilog and Chisel
  \item User and supervisor libraries in C for safe transaction management
  \end{itemize}
  \emph{Biologically-inspired Optical Flow on FPGA} \\
  Interfaced a camera with a Xilinx FPGA and implemented a biologically-inspired optical flow algorithm composed of Gabor filters of different orientations for real-time image processing in Verilog
}

%------------------------------------------------

\entry
{2006--2010}
{BS {\normalfont Electrical Engineering}}
{Boston University}
{Summa Cum Laude -- GPA 3.85/4.0
  \begin{itemize}
  \item Senior Design Project: PID-controlled Object Tracking System on FPGA and DSP
  \item \href{https://www.youtube.com/watch?v=S2LgUL5JLqQ}{FPGA Frogger in Verilog}
  \item 32-bit MIPS 5-stage pipelined CPU in Verilog
  \end{itemize}
}

%------------------------------------------------

\end{entrylist}

%----------------------------------------------------------------------------------------
%	WORK EXPERIENCE SECTION
%----------------------------------------------------------------------------------------

\section{experience}

%% \subsection{Internships}

\begin{entrylist}

%------------------------------------------------

\entry
{Summer 2011}
{Intel Corporation}
{Hudson, MA}
{\emph{Graduate Technical Intern} \\
  Tested and validated memory controllers Verilog/SystemVerilog RTL for an Intel server microprocessor
}

\entry
{Summer 2011}
{Intel Corporation}
{Hudson, MA}
{\emph{Graduate Technical Intern} \\
  Designed and optimized x86 assembly hashing functions while evaluating possible solutions for quality
}

%------------------------------------------------

\end{entrylist}

%----------------------------------------------------------------------------------------
%	AWARDS SECTION
%----------------------------------------------------------------------------------------

\section{awards}

\begin{entrylist}

%------------------------------------------------

\entry
{2012--2016}
{Space Technology Research Fellowship}
{NASA}
{Four year NASA fellowship titled \emph{Biologically-Inspired Hardware for Land/Aerial Robots}. This award has included yearly on-sites at NASA Jet Propulsion Lab.}

\entry
{2012}
{CELEST/CompNet Award}
{Boston University}
{Awarded at Boston University's Science Day for work titled \emph{Biologically-inspired Hardware for Autonomous Robots}}

\entry
{2010}
{Dean's Fellowship}
{Boston University}
{Full scholarship merit award for first year Ph.D students}

\entry
{2010}
{P. T. Hsu Memorial Award for Outstanding Senior Design Project}
{Boston University}
{Awarded to the best senior design project team}

\entry
{2006--2010}
{BU Engineering Scholar Award}
{Boston University}
{Undergraduate merit award with half-tuition scholarship}

%------------------------------------------------

\end{entrylist}

%----------------------------------------------------------------------------------------
%	INTERESTS SECTION
%----------------------------------------------------------------------------------------

%% \section{interests}

%% \textbf{professional:} hardware design, machine learning, open source software\\ \textbf{personal:} figure skating, skiing, biking, piano

%----------------------------------------------------------------------------------------
%	PUBLICATIONS SECTION
%----------------------------------------------------------------------------------------

\section{publications}

\printbibsection{inproceedings}{conference publications}

\printbibsection{article}{articles}

\printbibsection{misc}{patents and patent applications}

%----------------------------------------------------------------------------------------

\end{document}
