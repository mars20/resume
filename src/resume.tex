\documentclass[letterpage]{article}
\usepackage[letterpaper,margin=0.75in]{geometry}
\usepackage[scaled]{helvet}
\renewcommand*\familydefault{\sfdefault} %% Only if the base font of the document is to be sans serif
\usepackage[T1]{fontenc}
\usepackage{hyperref}
\pagestyle{empty}
%% \setlength{\tabcolsep}{0em}

\hypersetup {
  pdfauthor={Schuyler Eldridge},
  pdftitle={Resume -- Schuyler Eldridge},
}

\usepackage{enumitem}
\setlist[description]{leftmargin=1.5em,labelindent=0em}

\newcommand{\headerrow}[2]
{\begin{tabular*}{\linewidth}{l@{\extracolsep{\fill}}r}
	#1 &
	#2 \\
\end{tabular*}}
\newcommand{\CPP}
{C\nolinebreak[4]\hspace{-.05em}\raisebox{.22ex}{\footnotesize\bf ++}}


\begin{document}
\begin{center}

{\huge \textbf{Schuyler Eldridge}}\\
\href{mailto:schuyler.eldridge@gmail.com}{schuyler.eldridge@gmail.com}
$\bullet$ \url{http://people.bu.edu/schuye} $\bullet$ 914.382.1315
\end{center}
\hrule
\vspace{5pt}

\noindent
\begin{minipage}[t]{0.36\textwidth}
  %% \flushright
  \subsection*{Education}
  \vspace{-5pt}
  \begin{description}
  \item[Boston University] \hfill Boston, MA\\ \emph{PhD Computer
    Eng.} \hfill May 2016\\ Dean's Fellowship \hfill GPA 3.70/4.0
    \\NASA Space Technology Fellowship
  \item[Boston University] \hfill Boston, MA\\ \emph{BS Electrical
    Eng.} \hfill 2010\\ Summa Cum Laude \hfill GPA 3.85/4.0
  \end{description}

  \subsection*{Computer Skills}
  \vspace{-5pt}
  \begin{description}
  \item[Fluent Languages] \hfill\\ Bash, C, Chisel, \LaTeX, GNU Make, Matlab, Perl, Verilog, System Verilog
    \vspace{-5pt}
  \item[Proficient Languages] \hfill\\ \CPP, Java, Python, Assembly (x86, RISC-V), Scala, TCL
    \vspace{-5pt}
  \item[Operating Systems] \hfill\\ Linux, Windows
    \vspace{-5pt}
  \item[Tools] \hfill\\ Git, Emacs, Modelsim, Cadence RTL Compiler, Cadence SOC Encounter, Icarus Verilog, GTKWave, Inkscape
  \end{description}

  \subsection*{Relevant Coursework}
  \vspace{-5pt}
  \begin{itemize}\setlength\itemsep{0pt}
  \item Computer Organization, Architecture
  \item Advanced Digital Design in Verilog
  \item VLSI, VLSI Project
  \item Advanced Data Structures
  \item Digital Signal Processing
  %% \item Microprocessors
  \item Control Systems
  %% \item Signals and Systems
  \item Principles and Methods of Cognitive and Neural Modeling
  %% \item Learning Models
  \end{itemize}
  %% Computer Organization, Computer Architecture (EC513), Advanced
  %% Digital Design in Verilog (EC551), Microprocessors (EC450), VLSI
  %% (EC571), VLSI Project (EC772), Principles and Methods of Cognitive and
  %% Neural Modeling (CN510), Learning Models (CN570)

  %% \subsection*{Awards}
  \vspace{-5pt}
  %% \emph{NASA Space Tech. Research Fellowship}\\
  %% \emph{Dean's Fellowship}\\
  %% \emph{BU Engineering Scholar Award}

  \subsection*{Links}
  \vspace{-5pt}
  \href{http://www.github.com/seldridge}{GitHub://seldridge}\\
  \href{http://www.facebook.com/schuyler.eldridge}{Facebook://schuyler.eldridge}\\
  \href{http://www.linkedin.com/in/schuylereldridge}{LinkedIn://schuylereldridge}\\
  \href{http://www.twitter.com/theSchuyler}{Twitter://theSchuyler}

  \subsection*{Activities}
  \vspace{-5pt} Former nationally competitive Figure Skater---2007 US
  Nationals competitor, 2009 and 2010 US National Intercollegiate Team
  Champion representing Boston University

\end{minipage}\hfill
\begin{minipage}[t]{0.59\textwidth}
  \subsection*{Research}
  \vspace{-5pt}
  \begin{description}
  \item[Neural Network Accelerator Architectures] \hfill
    2012--Present\\ This research focuses on the design,
    implementation, and use of a \emph{multithreaded neural network accelerator} tightly integrated with a RISC-V microprocessor for acceleration of machine learning applications and to enable automatic microprocessor parallelization.
    \begingroup
    \renewcommand{\section}[2]{}
    \begin{thebibliography}{2014}
    \bibitem[2015]{eldridge2015a}
      \href{}{Eldridge, S., Waterland, A., Seltzer, M., Appavoo, J., and Joshi, A. Towards General-Purpose Neural Network Computing in {\em
          Parallel Architectures and Compilation Techniques (PACT)}.}
    \bibitem[2015]{eldridge2015}
      \href{http://people.bu.edu/schuye/files/barc2015-eldridge-paper.pdf}{Eldridge, S. and Joshi, A. Exploiting Hidden Layer Modular Redundancy for Fault-Tolerance in Neural Network Accelerators in \em{Boston Area Architecture Workshop (BARC)}}
    \bibitem[2014]{appavoo2014}
      \href{http://people.bu.edu/schuye/files/appavoo-neuroarch-2014.pdf}{Appavoo,
        J., Waterland, A., Zhao, K., Eldridge, S., Joshi, A., Seltzer,
        M., Homer, S. Programmable Smart Machines: A Hybrid
        Neuromorphic Approach to General Purpose Computation in {\em workshop on Neuromorphic Architectures (NeuroArch).}}
    \bibitem[2014]{eldridge2014}
      \href{http://people.bu.edu/schuye/files/glsvlsi2014-eldridge.pdf}{Eldridge,
        S., Raudies, F., Zou, D., and Joshi, A. Neural
        Network-based Accelerators for Transcendental Function
        Approximation. In {\em Great Lakes Symposium on VLSI (GLSVLSI)}.}
    \bibitem[2013]{eldridge2013}
      \href{http://people.bu.edu/schuye/files/approx-fpu-bic2013.pdf}{Eldridge,
        S., Raudies, F., and Joshi, A. Approximate Computation
        Using a Neuralized FPU. In {\em Brain Inspired Computing Workshop (BIC)}.}
    \end{thebibliography}
    \endgroup
    %% \item[Autonomous Navigation in Land and Aerial Robots] \hfill
    %%   2010--2012\\ In a joint project with Boston University's Center
    %%   for Computational Neuroscience and Neural Technologies, I
    %%   developed a hardware implementation of a high density optic flow
    %%   algorithm.  \begingroup \renewcommand{\section}[2]{}
    %%   \begin{thebibliography}{2014}
    %%   \bibitem[2014]{raudies2014}
    %%     \href{http://link.springer.com/article/10.1007/s10015-014-0153-1}{
    %%       Raudies, F., Eldridge, S., Joshi, A., and Versace, M. (2014).
    %%       Learning to navigate in a virtual world using optic flow and
    %%       stereo disparity signals. {\em Artificial Life and Robotics}.}
    %%   \end{thebibliography}
    %%   \endgroup
  \end{description}

  \subsection*{Work Experience}
  \vspace{-5pt}
  \begin{description}
  \item[NASA Jet Propulsion Lab] \hfill Summer 2013, 2014, 2015\\ Summer intern as part of a NASA Space Technology Research Fellowship working on fault-tolerance of and the design/implementation of neural network accelerators.
  \item[Intel Corporation] \hfill Summer 2010, 2011\\ Two internships working on the testing and validation of Haswell memory
    controllers (2011) and the design and optimization of hashing
    functions in x86 assembly for performance and quality (2010). \begingroup
    \renewcommand{\section}[2]{}
    \begin{thebibliography}{2011}
    \bibitem[2011]{gopal2011}
      \href{http://www.google.com/patents/US20130290285}{Gopal, V.,
        Guilford, J.~D., Eldridge, S., Wolrich, G.~M., Ozturk, E., and
        Feghali, W.~K. Digest generation. US Patent
        App. 13/995,236.}
    \end{thebibliography}
    \endgroup
  \item[Teaching Assistant / Guest Lecturer, Logic Design] \hfill
    2009--2011\\ Designed final projects, provided lab help, and
    delivered guest lectures on Verilog 2001 to undergraduate and
    graduate students.
  \end{description}

  \subsection*{Projects}
  \vspace{-5pt}
  \begin{description}
  \item[Capstone Project, Object Tracking System] \hfill Fall
    2009--Spring 2010
    %% Winner P.T. Hsu Best Project Award Designed a
    %% proportional/intergral/derivative control algorithm design and
    %% color threshold object detection Implemented on a custom PCB with
    %% a BlackFin DSP interfacing with a Xilinx FPGA
  \item[FPGA Frogger, Advanced Digital Design in Verilog]\hfill Fall
    2009
    %% Combinational pixel generation for VGA display, pseudo-random
    %% hardware number generation
  \item[32-bit MIPS CPU in Verilog, Computer Organization]\hfill Fall 2008
    %% 5-stage pipelined CPU running MIPS instruction subset with hazard
    %% detection / data forwarding
  \end{description}

  %%   \subsection*{Relevant Coursework}
  \vspace{-5pt}
  %%   Computer Organization (EC413), Computer Architecture (EC513), Advanced
  %%   Digital Design in Verilog (EC551), Microprocessors (EC450), VLSI
  %%   (EC571), VLSI Project (EC772), Principles and Methods of Cognitive and
  %%   Neural Modeling (CN510), Learning Models (CN570)
  %% \end{description}

\end{minipage}

\end{document}
